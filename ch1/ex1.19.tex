\documentclass[a4paper]{article}
\usepackage{amsmath}
\usepackage[T1]{fontenc}
\usepackage[scale=0.82]{geometry}
\usepackage{parskip}

\title{SICP - Exercise 1.19}
\author{Pratyush Nalam}
\date{July 7, 2016}

\begin{document}
	\maketitle
	
	The original transformation, $T_{pq}$ is given as
	
	\[
	T_{pq}(a,b)=
	\begin{cases}
	a\leftarrow a(p+q)+bq\\
	b\leftarrow bp+aq
	\end{cases}
	\]
	
	The new transformation $T_{p^\prime q^\prime}$ is defined as $T_{pq}(T_{pq})(a,b)$ -- basically the transformation is applied twice. Substituting this in the expression for $T_{pq}$, we get the following:
	
	\[
	T_{p^\prime q^\prime}(a,b)=
	\begin{cases}
	a\leftarrow (a(p+q)+bq)(p+q)+(bp+aq)q\\
	b\leftarrow (bp+aq)p+(a(p+q)+bq)q
	\end{cases}
	\]
	
	We expand the expression and simplify it to the following
	
	\[
	T_{p^\prime q^\prime}(a,b)=
	\begin{cases}
	a\leftarrow a(p^2+q^2+2pq+q^2)+b(2pq+q^2)\\
	b\leftarrow b(p^2+q^2)+a(2pq+q^2)
	\end{cases}
	\]
	
	Clearly, this means that
	
	\[
	p^\prime=p^2+q^2
	\]
	\[
	q^\prime=2pq+q^2
	\]
	
	This is the required solution and we use it to finish the given program.
	
\end{document}